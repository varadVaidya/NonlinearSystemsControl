\hfill\small{4 Jan 2024}
\vspace{0.5em}
\hrule
\vspace{-0.5em}
\section{Introduction to Nonlinear Systems}

In general, a nonlinear system can be defined with the following:
\[
    \begin{aligned}
        f &: \mathbb{R}^n \to \mathbb{R}^n \\
        \frac{dx}{dt} &= \dot{x} = f(t,x,u,d)  \\
        y &= g(t,x,u,d)
    \end{aligned}
\]
with the initial condition,
\[
    x(t_0) = x_0  
\]
where, \( x \in \mathbb{R}^n \) is the state, \( u \in \mathbb{R}^m \) is the control
input, \( y \in \mathbb{R}^p \) is the output, and \( d \in \mathbb{R}^q \) is the 
disturbance.

We can also write the control signal \(u\) as a function of the state \(x\) and some 
parameter vector \(\theta\). Thus, using this ``state feedback'' we have:
\[
    \begin{aligned}
        u &= h(t,x,\theta) \\
        \implies \dot{x} &= \bar{f} (t,x,\theta) \\
    \end{aligned}
\]
This free choice of \(\theta\) allows us to design the system to meet certain requirements, or
to optimize some performance metric etc.

One of the special case of nonlinear systems is called as autonomous systems, where the
system does not depend on time explicitly. Thus, we have:
\[
    \begin{aligned}
        \dot{x} &= f(x) \\
    \end{aligned}
\]
It is possible to convert a non-autonomous system to an autonomous system by adding a
new state variable \(z\), such that:
\[
    \dot{z} = 1   
\]
and incorporating the new variable \(z\), in the state vector, increasing the order of the 
system.
\begin{note}
    Note that, the state \(z\) and hence the state \(x\), is unbounded with time in this 
    conversion.
\end{note}  

In the case of the linear systems, we have the following:
\[
    \begin{aligned}
        \dot{x} & = A(t) x + B(t) u \\
        y &= C(t) x + D(t) u 
    \end{aligned}  
\]
with the solution of the above system being:
\[
    x(t) = \Phi(t,t_0) x(t_0) + \int\limits_{t_0}^{t} \Phi(t,\tau) B(\tau) u(\tau) d\tau
\]
with the state transition matrix \(\Phi(t,\tau)\). The state transition matrix simplifies
even further under the case of LTI systems as:
\[
    \Phi(t,\tau) = e^{A(t-\tau)}  
\]

In genreal the solution of the nonlinear system cannot be wriiten in a closed form. However, there are
certain observations that are present in the nonlinear systems, that are not present in the linear systems.

\subsection{Observations in Nonlinear Systems (vs Linear Systems)}

\subsubsection{Equillibrium}
In general the eqillibrium of any system \(\dot{x} = f(x)\) is defined as the point \(x^{\star}\) where:
\[
    \left. f(x) \right|_{x^{\star}} = 0  \iff \left. \dot{x} \right|_{x^{\star}} = 0 
\]
\begin{example}[Linear Systems]
    Consider the case of the linear system,
    \[
        \dot{x} = A x
    \]
    The eqillibrium of the above system is given by:
    \[
        Ax = 0 \implies x \in \mathcal{N}(A)
    \]
    where, \(\mathcal{N}(A)\) is the null space of the matrix \(A\).
\end{example}

No such existence of the eqillibrium is guaranteed in the case of the nonlinear systems. 
Thus, we can have a nonlinear system with no eqillibrium, or multiple eqillibria.
\begin{example}[Nonlinear Systems]
Here are some examples of nonlinear systems with no, multiple or infiite equillibria.
    \begin{enumerate}
        \item
        \[
            \dot{x} = Ax \quad \quad A = \begin{bmatrix}
                0 & 1 \\
                -1 & 0  
            \end{bmatrix}
        \]
        The above system has infiite equillibrium points, given by:
        \[
            x^{\star} = \alpha \begin{bmatrix}
                0 \\ 1
            \end{bmatrix} \quad \forall \alpha \in \mathbb{R}
        \]
        
        \item
        \[
            \dot{x} = x^{2} + a \quad a > 0 
        \]
        This system has no eqillibrium point.

        \item
        \[
            \dot{x} = x^{2} - a \quad a > 0
        \]
        This system has two eqillibrium points, given by \(x^{\star} = \pm \sqrt{a}\).
    \end{enumerate}
\end{example}

\subsubsection{Finite Escape Time}
In general, the solution of the nonlinear system is not guaranteed to be bounded for all time, and 
can escape to infinity in a finite time. This is not possible in the case of the linear systems.
\begin{example}[Linear Systems]
    Consider the case of the linear system,
    \[
        \dot{x} = x \implies x(t) = e^{t} x_0
    \]
    The solution approaches infinity as \(t \to \infty\), in an asymptotic manner, but never reaches
    infinity in a finite time.
\end{example}\vspace{1em}
But, in the case of nonlinear systems, the solution can escape to infinity in a finite time.
\begin{example}[Nonlinear Systems]
    Consider the case of the nonlinear system,
    \[
        \dot{x} = 1 + x^{2} \quad x \in \mathbb{R} 
    \]
    Integrating the system, we get:
    \[
        \begin{aligned}
            \frac{dx}{1+x^{2}} = dt \\
            \implies \left. \tan^{-1} x \right|_{x_0}^{x(t)} &= \left. t \right|_{0}^{t} \\
            \implies x(t) &= \tan(t + \tan^{-1}(x_0))
        \end{aligned}
    \]
    Let \(x_0 = 0\), then we have:
    \[
        x(t) = \tan(t)
    \]
    The solution of the above system, approaches \(\infty\) as \(t \to \frac{\pi}{2}\). Since the solution
    goes unbounded in a finite time, we say that the system has a finite escape time.  
\end{example}

\subsection{Uniqueness of Solution}
In general, the solution of the nonlinear system is not unique. 
\begin{example}[Uniqueness of Solution]
    Consider the following system:
    \[
        \dot{x} = \sqrt{2} ,x(0) = 0 , x \in \mathbb{R}  
    \]
    For this system,
    \[
        x \equiv 0 \text{ is a solution}
    \]
    But, we can also have the following solution:
    \[
        x_\alpha = \begin{dcases}
            \frac{(t - \alpha)^{2}}{4}, & t \geq \alpha, \alpha > 0 ;\\
            0 , & t< \alpha  ;\\
        \end{dcases}
    \]
    for each \(\alpha \in \mathbb{R}\), with the same initial conditions, we have infiite solutions 
\end{example}
Thus, the solution of the nonlinear system is not unique.
This is a big problem, as we cannot predict the behaviour of the system, which can lead to break
down of the system.